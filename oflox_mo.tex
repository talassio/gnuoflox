%  ############################################################################
%  #    This file is part of GNU Oflox                                        #
%  #    Copyright (C) 2006, 2007, 2008, 2009 by Andres Sajo                   #
%  #    talassio-at-gmail-dot-com                                             #
%  #                                                                          #
%  #    This program is free software; you can redistribute it and/or modify  #
%  #    it under the terms of the GNU General Public License as published by  #
%  #    the Free Software Foundation; either version 3 of the License, or     #
%  #    (at your option) any later version.                                   #
%  #                                                                          #
%  #    This program is distributed in the hope that it will be useful,       #
%  #    but WITHOUT ANY WARRANTY; without even the implied warranty of        #
%  #    MERCHANTABILITY or FITNESS FOR A PARTICULAR PURPOSE.  See the         #
%  #    GNU General Public License for more details.                          #
%  #                                                                          #
%  #    You should have received a copy of the GNU General Public License     #
%  #    along with this program; if not, write to the                         #
%  #    Free Software Foundation, Inc.,                                       #
%  #    59 Temple Place - Suite 330, Boston, MA  02111-1307, USA.             #
%  ############################################################################
%
%  Universidad Sim\'on Bol\'ivar
%  Coordinaci\'on de Matematicas
%  Departamento de C\'omputo Cient\'ifico y Estad\'istica
%
% oflox_mo LaTeX output template
% Version \FMCversion
% 05/07/08, 20/08/2009
%
\documentclass[11pt]{article}
\usepackage{graphics} % Necesary
\usepackage{xcolor} % Necesary
\usepackage{amsfonts} % Necesary
\usepackage{amsmath} % Necesary
\usepackage[letterpaper,left=1.5cm,right=1cm, top=1cm, bottom=1cm]{geometry}

% Graphics format (this goes right in to \includegraphics(...))
\newcommand{\FMCimageformat}{ps}

% space between tables
\newcommand{\FMCtblseparation}{1.0em}
\newcommand{\FMCvspace}{1.0em}

%%%%%%%%%%%%%%%%%%%%%%%%%%%%%%%%%%%%%%%%%%%%%%%%%%%%%%%%%%%%%
%% LANGUAGE DEFINITIONS

% Spanish version
\newcommand{\FMCiterlabeles}{Iteraci\'on\  \theiterationcnt\ (\FMCproblem)\\[1.0ex]}
% English version
\newcommand{\FMCiterlabelen}{Iteration\  \theiterationcnt\ (\FMCproblem)\\[1.0ex]}

% Captions of \FMCinit
%  1) The problem description label
%  2) The iterative initial label

% Spanish version
\newcommand{\FMCproblemlabeles}{Descripci\'on del problema\ (\FMCproblem)\\[1.0ex]}
\newcommand{\FMCinitiallabeles}{Descripci\'on de la iteraci\'on inicial\ (\FMCproblem)\\[1.0ex]}
% English version
\newcommand{\FMCproblemlabelen}{Description of the problem\ (\FMCproblem)\\[1.0ex]}
\newcommand{\FMCinitiallabelen}{Description of the initial iteration\ (\FMCproblem)\\[1.0ex]}

% Caption of \FMCsolution
% Spanish version
\newcommand{\FMCfinallabeles}{Soluci\'on del problema\ (\FMCproblem)\\[1.0ex]}
% English version
\newcommand{\FMCfinallabelen}{Solution of the problem\ (\FMCproblem)\\[1.0ex]}

% Block description format
% Spanish version
\newcommand{\FMCcaptiones}{\footnotesize Iteraci\'on\  \theiterationcnt:\ }
% English version
\newcommand{\FMCcaptionen}{\footnotesize Iteration\  \theiterationcnt:\ }

%% END LANGUAGE DEFINITIONS
%%%%%%%%%%%%%%%%%%%%%%%%%%%%%%%%%%%%%%%%%%%%%%%%%%%%%%%%%%%%%

% Tunable variables:
\newcommand{\FMCref}[1]{\arabic{FMC#1cnt}}
\newcommand{\FMClabel}[1]{%
        \newcounter{FMC#1cnt} %
        \setcounter{FMC#1cnt}{\theiterationcnt}%
}
\newcommand{\FMCproblem}{UNDEFINED}
\newcommand{\FMCname}[1]{ \renewcommand{\FMCproblem}{#1} }

\newcommand{\FMCversion}{1.12}

%% Counters:
\newcommand{\inc}[1]{\stepcounter{#1}}
\newcommand{\dec}[1]{\addtocounter{#1}{-1}}
\newcounter{iterationcnt}
\setcounter{iterationcnt}{0}

% Reset counter by calling \FMCinit
\newcommand{\newFMC}{\setcounter{iterationcnt}{0}}

%% Internals

% Artificial markup symbol
\newlength{\FMCspaceasymbol}
\newcommand{\FMCasymbol}{$_{\text{a}}$}
\settowidth{\FMCspaceasymbol}{\FMCasymbol}

% Big-M markup symbol
\newlength{\FMCspaceMsymbol}
\newcommand{\FMCMsymbol}{$_{\text{M}}$}
\settowidth{\FMCspaceMsymbol}{\FMCMsymbol}

\newcommand{\FMCdescription}[2]{%
% Initial problem description
        \begin{center}%
        \rule{\linewidth}{0.333mm}\\%
        \FMCproblemlabel%
        \scalebox{#1}{\includegraphics{\FMCproblem.\FMCimageformat}}\\[\FMCvspace]%
        \input{\FMCproblem.tex}\\[\FMCvspace]%
        {\footnotesize #2 } %
        \rule{\linewidth}{0.333mm}%
        \end{center}%
}

\newcommand{\FMCinit}[2]{%
        \begin{center}%
        \rule{\linewidth}{0.333mm}\\%
        \setcounter{iterationcnt}{0}%
        \FMCinitiallabel%
        \scalebox{#1}{\includegraphics{\FMCproblem_\theiterationcnt.\FMCimageformat}}\\[\FMCvspace]%
        \input{\FMCproblem_\theiterationcnt.tex}\\[\FMCvspace]%
        % [moved to\FMCiter action item 5: 20/08/2009] \input{\FMCproblem_TV_\theiterationcnt.tex}\\[\FMCvspace]%
        { %\FMCcaption
        #2 } %
        \rule{\linewidth}{0.333mm}%
        \end{center}%
}

\newcommand{\FMCiter}[2]{%
        \begin{center}%
        \rule{\linewidth}{0.333mm}\\%
        \inc{iterationcnt}%
        \FMCiterlabel%
        \dec{iterationcnt}%
        \input{\FMCproblem_TV_\theiterationcnt.tex}\\[\FMCvspace]%
        \inc{iterationcnt}%
        \hfill \scalebox{#1}{\includegraphics{\FMCproblem_\theiterationcnt.\FMCimageformat}}%\FMCtblseparation
        \hfill \scalebox{#1}{\includegraphics{\FMCproblem_\theiterationcnt a.\FMCimageformat}} \hfill \null \\[\FMCvspace]%
        \input{\FMCproblem_\theiterationcnt.tex}\\[\FMCvspace]%
        % [moved to\FMCiter action item 5: 20/08/2009] \input{\FMCproblem_TV_\theiterationcnt.tex}\\[\FMCvspace]%
        {\FMCcaption #2 } %
        \rule{\linewidth}{0.333mm}%
        \end{center}%
}

\newcommand{\FMCsolution}[2]{%
        \begin{center}%
        \rule{\linewidth}{0.333mm}\\%
        \inc{iterationcnt}%
        \FMCfinallabel%
        \dec{iterationcnt}%
        \input{\FMCproblem_TV_\theiterationcnt.tex}\\[\FMCvspace]%
        \inc{iterationcnt}%
        \scalebox{#1}{\includegraphics{\FMCproblem_\theiterationcnt.\FMCimageformat}}\\[\FMCvspace]%
        \input{\FMCproblem_\theiterationcnt.tex}\\[\FMCvspace]%
        { %\FMCcaption
        #2 } %
        \rule{\linewidth}{0.333mm}%
        \end{center}%
}

%%%%%%%%%%%%%%%%%%%%%%%%%%%%%%%%%%%%%%%%%%%%%%%%%%%%%%%%%%


